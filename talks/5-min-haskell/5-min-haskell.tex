% WARNING! you need latex-beamer installed on your system
% If you are using darwinports on MacOS X, this is already
% included.

%\documentclass[compress,10pt,a4paper]{beamer}
\documentclass[compress]{beamer}

% comment this if you want to get rid of the little points
\useoutertheme[subsection=false]{miniframes}
\usecolortheme{dove}


\definecolor{marroon}{rgb}{0.5,0,0}
\newcommand\highlightNew{\color{red}}
\newcommand\highlightOld{\color{marroon}}

%\usepackage[latin1]{inputenc}
\usepackage{graphicx}
\usepackage{url}
\usepackage{listings}

\lstset{language=Haskell,
        basicstyle=\footnotesize\tt,
        commentstyle=\tt}

\begin{document}

\title{Why I use Haskell}
\author{Eric Kow}
\institute{\url{http://erickow.com/talks/5-min-haskell.pdf}}
\date{2011-05-17}

\begin{frame}
\maketitle
\end{frame}

% -------------------------------------------------------------------------
\begin{frame}
\frametitle{Things I care most about}

\begin{enumerate}
\item Problem solving/identification:\\
      \small{How do I solve X?  What is X in the first place?}\\
\item Correctness\\
      \small{How do I solve X with as few bugs as possible?}\\
\item Clarity\\
      \small{Will Future-Eric understand this code? Will my colleagues?}\\
\end{enumerate}

\end{frame}


% -------------------------------------------------------------------------
\begin{frame}
\frametitle{Types}

\begin{enumerate}
\item Problem solving: think about why types I need\\
\small{Easier for me than thinking about object hierarchies}
\item Problem solving: sum types = thinking in terms of cases 
\item Clarity - types == documentation\\
      \small{I can sort of tell what a function does by its type signature}
\item Correctness\\
      \small{Compiler catches when I contradict myself}
\end{enumerate}
\end{frame}

% -------------------------------------------------------------------------
\begin{frame}[fragile]
\frametitle{Types 2}
\begin{block}{Example}
\begin{lstlisting}
data Compound = And Compound Compound
              | Or Compound Compound
              | Product Compound Compound
              | Not Compound
              | Leaf Unitary
   deriving (Show, Eq)
\end{lstlisting}
\end{block}

Code that works with Compound must account for all cases.
\end{frame}


% -------------------------------------------------------------------------
\begin{frame}
\frametitle{Higher order functions}

A higher order function is a function that takes function(s)
as an argument

\begin{enumerate}
\item Easier to reuse my code!
\item Very general purpose traversals
\end{enumerate}

\end{frame}

% -------------------------------------------------------------------------
\begin{frame}[fragile]

\begin{block}{Higher order function example}
\begin{lstlisting}
traverse :: (Unitary -> Compound) -> Compound -> Compound
traverse f (And x y) = And (traverse f x) (traverse f y)
traverse f (Or x  y) = Or  (traverse f x) (traverse f y)
traverse f (Product x y) = Product (traverse f x)
                                   (traverse f y)
traverse f (Not x) = Not (traverse f x)
traverse f (Leaf x) = f x
\end{lstlisting}
\end{block}
\end{frame}

% -------------------------------------------------------------------------
\begin{frame}
\frametitle{Immutability}
\begin{enumerate}
\item Solve small problems (functions)
\item Combine them with powerful tools (higher order functions)
\item Not worry about side effects!
\end{enumerate}

\end{frame}


\end{document}
